\newcommand{\myline}[1]{\underline{\hspace{#1}}}
\newcounter{problems}
\newcounter{questions}[problems]
\newcounter{subquestions}[questions]
\newcommand{\problem}[2]{\stepcounter{problems} {\Large \bf \noindent Problem \arabic{problems}: #1 \emph{(#2 pts)}} \\[.4cm]}
\newcommand{\question}[2]{\stepcounter{questions} \noindent{\emph{\Large Question (\alph{questions}): #1 (#2 pts) }\\[.4cm]}}
\newcommand{\subquestion}[2]{\stepcounter{subquestions} \noindent{\emph{\Large Subquestion (\roman{subquestions}): #1 (#2 pts) }\\[.4cm]}}

% Some standard centering and italicization of text.
\newcommand{\frontrowcenter}[1]{\begin{center}{\em \Large  #1  }\end{center}}

% A blank page
\newcommand{\blankpage}{
\clearpage
\vspace*{\fill}
	\begin{minipage}{\textwidth}
		\hspace{.5in} \Large \textbf{THIS PAGE INTENTIONALLY LEFT BLANK}\\
	\end{minipage}
\vfill % equivalent to \vspace{\fill}
\clearpage
}

\newcommand{\answerspace}[1]{
	\begin{center}
		\textbf{BEGIN YOUR ANSWER BELOW THIS LINE} \\ \hrulefill \vspace{#1} \\ \hrulefill
	\end{center}
}

\newcommand{\answerspacefullpage}{
	\begin{center}
		\textbf{BEGIN YOUR ANSWER BELOW THIS LINE} \\ \hrulefill \pagebreak
	\end{center}
}


\newcommand{\additionalanswerspace}[1]{
	\begin{center}
		\textbf{CONTINUE YOUR ANSWER FOR #1 BELOW THIS LINE } \\ \hrulefill \vspace{#1} \\ \hrulefill
	\end{center}
}

\newcommand{\additionalanswerspacefullpage}{
	\begin{center}
		\textbf{CONTINUE YOUR ANSWER BELOW THIS LINE} \\ \hrulefill \pagebreak
	\end{center}
}

\newcommand{\freespace}[1]{
	\begin{center}
		\large \textbf{SCRAP SPACE BELOW} \\ 
		\hrulefill
		\pagebreak 
	\end{center}
}

\newcommand{\silentanswerspace}[1]{
	\\ \hrule \vspace{#1} \hrule 
}


\newcommand{\notespage}{
	\pagebreak
	\begin{center}
		\Large \textbf{SCRAP PAPER} \\ \hrulefill
	\end{center}
	\pagebreak
}

\newcommand{\showyourwork}{
	\begin{center}
		\large \textbf{SHOW YOUR WORK BELOW} \\ 
		\hrulefill
		\pagebreak
	\end{center}
}

\newcommand{\justifywork}{
	\begin{center}
		\large \textbf{JUSTIFY YOUR ANSWERS BELOW} \\ 
		\hrulefill
		\pagebreak
	\end{center}
}

% Space for T/F:
\newcommand{\tfline}{\myline{.5cm}}

% For sets:
\newcommand{\curlybraces}[1]{\ensuremath{\lbrace #1 \rbrace}}

% For cardinalities:
\newcommand{\crd}[1]{\ensuremath{ \big \vert #1 \big \vert }}

\newcommand{\yesno}{{\footnotesize YES / NO}}
\newcommand{\truefalse}{{\normalsize  \bf TRUE / FALSE}}

% \item environments coupled with a line at the end, for students to write T and F in.
\newcommand{\tfitem}[1]{\item #1 \null\hfill \framebox(25,25){} \\ \hdashrule{0.95\textwidth}{1pt}{2pt}}
\newcommand{\setitem}[1]{\tfitem{$\curlybraces{#1}$} }
\newcommand{\lineitem}[1]{\item #1 \null \hfill \myline{1in}}

% Some circles for people to fill in,
\newcommand{\whitecircle}[1]{\tikz[baseline=-0.5ex]\draw[black, radius=#1] (0,0) circle ;}
%\newcommand{\blackcircle}[2][black,fill=black]{\tikz[baseline=-0.5ex]\draw[black,radius=#1] (0,0)}
\newcommand{\blackcircle}[1]{\tikz\draw[black,fill=black,radius=#1] (0,0) circle (.5ex);}%